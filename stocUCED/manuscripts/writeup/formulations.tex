\documentclass[11pt,a4paper]{article}
\usepackage[utf8]{inputenc}
\usepackage{amsmath}
\usepackage{amsthm}
\usepackage{amsfonts}
\usepackage{amssymb}
%\usepackage{algorithmic}
%\usepackage{algorithm}
\usepackage{hyperref}
%\usepackage[ruled,linesnumbered,procnumbered]{algorithm2e}

\makeatletter
\let\original@algocf@latexcaption\algocf@latexcaption
\long\def\algocf@latexcaption#1[#2]{%
  \@ifundefined{NR@gettitle}{%
    \def\@currentlabelname{#2}%
  }{%
    \NR@gettitle{#2}%
  }%
  \original@algocf@latexcaption{#1}[{#2}]%
}
\makeatother

\usepackage{xcolor}		% for adding colors to the text

\usepackage[footnotesize, center]{caption}	
\usepackage{subcaption}
%\usepackage{subfigure}
\usepackage{booktabs}
\usepackage{multirow}
\usepackage{breakurl}	% fixes the problem of breaking long url into lines inside the bibliography (IMP: must be below the hyperref package)
\hypersetup{
    bookmarks=true,		% show bookmarks bar?
    colorlinks=true,		% false: boxed links; true: colored links
    linkcolor=blue,		% color of internal links
    citecolor=blue,		% color of links to bibliography
    filecolor=blue,		% color of file links
    urlcolor=blue,		% color of external links
    bookmarksopen=true,
    breaklinks=true,
}
%\usepackage{natbib}			% bibliography
%\setlength{\bibsep}{0pt}

\usepackage[title]{appendix}

\usepackage{longtable}

% Page Setup

% Margins
	% MS Word (Default):		[margin=0.98in]
	% MS Word (Narrow):		[margin=0.5in]
\usepackage[margin=2cm]{geometry}
\setlength{\parskip}{\medskipamount}

\usepackage{setspace}
%\onehalfspacing
\setstretch{1.15}

\usepackage{indentfirst}			% indent the first paragraph
\allowdisplaybreaks				% break if equations take too much vertical space

% Section Style
\makeatletter
\def\@seccntformat#1{\csname the#1\endcsname.\quad}		% put a fullstop after section numbers
\makeatother

% Optional packages
\usepackage{xfrac}	% nicer fractional symbols (e.g., \sfrac{1}{2})
\usepackage{fancyhdr}	% adding headers / footers

% MACROS

% mathematical
\newcommand{\mathbold}[1]{\boldsymbol{#1}}		% for adding bold greek letters
\newcommand{\binary}{\{ 0, 1 \} }

\newcommand{\parentheses}[1]{\left( #1 \right) }	% auto-size parentheses
\newcommand{\brackets}[1]{\left[ #1 \right] }		% auto-size brackets
\newcommand{\curly}[1]{\left\{ #1 \right\} }		% auto-size curly brackets

	% Nilay Noyan's commenting macros
	% commentbox (adds black comments in a frame box)
	% comment (adds red comments as a footnote)

\newcounter{commentcounter}
\setcounter{commentcounter}{1}

\setlength{\fboxsep}{5pt}
\setlength{\fboxrule}{0.1pt}

\newcommand{\commentbox}[1]{
% \hfill \newline \noindent
%	\framebox[\textwidth]{
%		\parbox{0.98\textwidth}{
%			\footnotesize{
%				\texttt{\textcolor{black}{(C.\arabic{commentcounter})~#1\hfill}}}}}
%	\addtocounter{commentcounter}{1} 
	}

\long\def\symbolfootnote[#1]#2{\begingroup\def\thefootnote{\fnsymbol{footnote}}\footnote[#1]{#2}\endgroup}

\newcommand{\comment}[2]{{\footnotesize\texttt{\textcolor{red}{(C.\arabic{commentcounter})}}\symbolfootnote[4]{\texttt{\textcolor{red}
        {(C.\arabic{commentcounter}) [#1]: ~#2}}}}\addtocounter{commentcounter}{1}}
%\newcommand{\comment}[2]{}

% theorems
% the subtheorem environment is used to generate theorem numbers 1a, 1b, etc.
% Source: http://tex.stackexchange.com/questions/43346/how-do-i-get-sub-numbering-for-theorems-theorem-1-a-theorem-1-b-theorem-2
\makeatletter
\newenvironment{subtheorem}[1]{%
  \def\subtheoremcounter{#1}%
  \refstepcounter{#1}%
  \protected@edef\theparentnumber{\csname the#1\endcsname}%
  \setcounter{parentnumber}{\value{#1}}%
  \setcounter{#1}{0}%
  \expandafter\def\csname the#1\endcsname{\theparentnumber\alph{#1}}%
  \ignorespaces
}{%
  \setcounter{\subtheoremcounter}{\value{parentnumber}}%
  \ignorespacesafterend
}
\makeatother
\newcounter{parentnumber}
% end of subtheorem environment

\newtheorem{theorem}{\bf Theorem}
\newtheorem{lemma}{\bf Lemma}
\newtheorem{proposition}{\bf Proposition}
\newtheorem{corollary}{\bf Corollary}
\newtheorem{definition}{\sc Definition}
\newtheorem{fact}{\bf Fact}
\newtheorem{claim}{\sc Claim}
\newtheorem{case}{\sc Case}
\newtheorem{observation}{\sc Observation}
\renewcommand{\qedsymbol}{\hfill \tiny$\blacksquare$}		% symbol for proof environment
\renewcommand{\proofname}{\textnormal{\textbf{Proof.}}}	% title in the proof environment

\newtheoremstyle{mytheoremstyle} % name
    {\topsep}                    % Space above
    {\topsep}                    % Space below
    {}                   % Body font
    {}                           % Indent amount
    {\scshape}                   % Theorem head font
    {.}                          % Punctuation after theorem head
    {.5em}                       % Space after theorem head
    {}  % Theorem head spec (can be left empty, meaning ‘normal’)
\theoremstyle{mytheoremstyle}
\newtheorem{example}{Example}

\newtheoremstyle{myassumptionstyle} % name
    {\smallskipamount}                    % Space above
    {0}                    % Space below
    {}                   % Body font
    {}                     	% Indent amount
    {\upshape}              % Theorem head font
    {.}                          	% Punctuation after theorem head
    {.5em}                      % Space after theorem head
    {}  				% Theorem head spec (can be left empty, meaning ‘normal’)
\theoremstyle{myassumptionstyle}
\newtheorem{assumption}{\bf A\ignorespaces}
\newtheorem{remark}{\bf Remark}

\DeclareMathOperator*{\argmin}{arg\,min} 
\DeclareMathOperator*{\argmax}{arg\,max} 


% narrative
\newcommand{\ie}{\textit{i.e.}}		% id est, that is to say
\newcommand{\ex}{\textit{ex.}}		% example
\newcommand{\eg}{\textit{e.g.}}	% exempli gratia, for the sake of example

\newcommand{\st}{\text{subject to:}\qquad}	% subject to
\newcommand{\mathbi}[1]{\boldsymbol{#1}}	% \boldsymbol{ any character } // makes both italic and bold

\newcommand{\cplex}{\texttt{CPLEX}}

\newcommand{\question}[1]{\vspace{\baselineskip}\noindent\pdfbookmark{Question #1}{Question #1}\textbf{\large{Question #1}} \normalsize\medskip\newline}
\newcommand{\qpart}[1]{\indent\textbf{#1)}}	% i.e., a), b), ...

\newcommand{\inlinecomment}[1]{{\color[rgb]{0.13,0.57,0.4} \textbf{#1}}}

% algorithmic
\newcommand{\np}{$\mathcal{NP}$}	% e.g., as in NP-hard

\usepackage{enumitem}	% for aligned descriptions
\usepackage{calc} 		% for aligned descriptions

\setenumerate{
itemsep=0pt,
partopsep=0pt,
parsep=0pt,
topsep=0pt,
labelindent=4pt,
font=\normalfont
}

\setdescription{
itemsep=0pt,
partopsep=0pt,
parsep=0pt,
topsep=0pt,
labelindent=4pt,
font=\normalfont
}

\setitemize{
itemsep=0pt,
partopsep=0pt,
parsep=0pt,
topsep=0pt,
labelindent=4pt,
font=\normalfont
}


\setlength{\textheight}{23cm} %{23cm}
\setlength{\topmargin}{-2cm}
\setlength{\textwidth}{17.5cm} \setlength{\oddsidemargin}{-0.5cm}
\setlength{\evensidemargin}{-0.5cm}

\setlength{\parindent}{0pt}

\newcommand{\gap}{\vspace{5pt}}
\newcommand{\epc}{\hspace{1pc}}

\newcommand{\onebld}{{\bf 1}}
\newcommand{\wt}{\widetilde}
\newcommand{\wh}{\widehat}

\newcommand{\E}{{\rm I\!E}}
\newcommand{\IP}{{\rm I\!P}}
\newcommand{\D}{{\rm I\!D}}
\newcommand{\pmat}[1]{\begin{pmatrix} #1 \end{pmatrix}}
\newcommand{\us}[1]{{\color{black}#1}}
\newcommand{\ssbs}[1]{{\color{blue}#1}}
\bibliographystyle{plain}

%% Semih's additions %%
\usepackage{mathtools}
%% end of semih's additions %%


\title{\bf Taming the Duck: Can Stochastic Programming Help?}
\date{}
%\author{Harsha Gangammanavar\\Engineering Management, Information, and Systems\\Southern Methodist University, \\ Dallas, TX 75275}

\begin{document}
\pagenumbering{gobble}
\maketitle

\vspace*{-1.3cm}

\section{Formulations}

\subsection{Decision Variables and Parameters}

Let $\mathcal{G}$ represent the set of generators, and $\mathcal{T}$ denote the set of discretized time periods. The set of buses is denoted with $\mathcal{B}$ and the set of lines is denoted with $\mathcal{L}$. The generators that are located at a particular bus $j$ are represented with the set $\mathcal{G}_j$. The following variables are defined for each generator $g \in \mathcal{G}$, and period $t \in \mathcal{T}$:

\begin{description} \setlength{\itemsep}{-1pt}
\item{$x_{gt}$:} 1 if generator $g$ is operational at $t$, 0 otherwise,
\item{$s_{gt}$:} 1 if generator $g$ becomes online at $t$, 0 otherwise,
\item{$z_{gt}$:} 1 if generator $g$ becomes offline at $t$, 0 otherwise,
\item{$G_{gt}$:} Production level of generator $g$ at $t$, 
\end{description}

For each bus $j\in \mathcal{B}$ and period $t \in \mathcal{T}$, we define the following variables: 
\begin{description}
\item{$\theta_{jt}$:} Voltage angle at bus $j$ at $t$,
\item{$L_{jt}$:} Amount of load shed at bus $j $ at $t$.
\item{$O_{jt}$:} Amount of over-generation at bus $j$ at $t$.
\end{description}
The latter two variables are penalized in the objective with the coefficients $L^{penalty}$ and $O^{penalty}$, respectively. 

Each generator $g \in \mathcal{G}$ is characterized by the following parameters:
\begin{description} \setlength{\itemsep}{-1pt}
\item{$G_g^{\max}$:} Maximum generation capacity, 
\item{$G_g^{\min}$: } Minimum generation requirement when the generator is online, 
\item{$\Delta G_g^{\max}$: } Ramp up limit, 
\item{$\Delta G_g^{\min}$:} Ramp down limit,
\item{$UT_g$:} Minimum required uptime before the generator can become offline, 
\item{$DT_g$:} Minimum required downtime before the generator can become online,
\item{$c_g^{gen}$:} Generation cost,
\item{$c_g^{start}$:} Start up cost,
\item{$c_g^{noload}$:} No load cost.
\end{description}

The problem is defined over a topology that is defined by the following characteristics:
\begin{description} \setlength{\itemsep}{-1pt}
\item{$F^{\min}_{ij,t}$ / $F^{\max}_{ij,t}$:} Flow lower / upper limits over line $(i,j) \in \mathcal{L}$ at $t$,
\item{$B_{ij}$:} Susceptance of line $(i,j) \in \mathcal{L}$.
\end{description}

Finally, each bus $j\in \mathcal{B}$ faces a demand $D_{jt}$ at period $t \in \mathcal{T}$. 


\subsection{Unit Commitment}

\begin{subequations}
\begin{align}
\begin{split}
\min \quad \sum_{g \in \mathcal{G}} \sum_{ t \in \mathcal{T} } \left(c^{gen}_{g} G_{gt} + c^{start}_g s_{gt} + c^{noload}_g x_{gt} \right) + \sum_{j \in \mathcal{B}} \sum_{ t \in \mathcal{T} } \left( L^{penalty} L_{jt} + O^{penalty} O_{jt} \right)
\end{split}
\\ 
\text{s.t.} \qquad & \parbox{3cm}{\it \small State Equations}  
\begin{dcases}
x_{gt} - x_{gt-1}  = s_{gt} - z_{gt},  & \forall g \in \mathcal{G}, \, t \in \mathcal{T}, \label{con:state_transition}
\end{dcases}  \\ 
%
& \parbox{3cm}{\it \small Minimum Up/Downtime Restrictions}
\begin{dcases}
 \sum_{j=t-UT_g+1}^{t-1} s_{gt} \leq x_{gt}, & \forall g \in \mathcal{G},\, t \in \mathcal{T}, 
%\label{con:min_uptime} 
\\ 
 \sum_{j=t-DT_g}^{t} s_{gt} \leq 1-x_{gt}, & \forall g \in \mathcal{G}, \, t \in \mathcal{T}, 
%\label{con:min_downtime}
\label{con:min_updown}
\end{dcases}
\\
%
& \parbox{3cm}{\it \small Generation Limits}
\begin{dcases}
G_g^{\min} x_{gt} \leq G_{gt} \leq G_g^{\max} x_{gt}, & \forall g \in \mathcal{G}, \, t \in \mathcal{T}, \label{con:gen_limits} 
\end{dcases} \\
% 
& \parbox{3cm}{\it \small Ramping Limits} 
\begin{dcases} 
 \Delta G_g^{\min} \leq G_{gt} - G_{gt-1} \leq \Delta G_g^{\max}, & \forall g \in \mathcal{G}, \, t \in \mathcal{T}, \label{con:ramping} 
\\ 
\end{dcases} \\
%
& \parbox{3cm}{\it \small Flow Limits} 
\begin{dcases} 
F^{\min}_{ij,t} \leq F_{ij,t} \leq F_{ij,t}^{\max}, & \forall (i,j) \in \mathcal{L}, t \in \mathcal{T}, \\ 
\end{dcases} 
\\
& \parbox{3cm}{\it \small Flow Balance} 
\begin{dcases} 
\sum_{i \in \mathcal{B} : (i,j) \in \mathcal{L}} F_{ij,t} - \sum_{i \in \mathcal{B} : (j,i) \in \mathcal{L}} F_{ji,t} + \sum_{g \in \mathcal{G}_j} G_{gt} + L_{jt} - O_{jt} = D_{jt}, & j \in \mathcal{B}, t \in \mathcal{T}, \\ 
\end{dcases} 
\\
& \parbox{3cm}{\it \small Power Flow \newline Approximation} 
\begin{dcases} 
F_{ij,t} = B_{ij} (\theta_{it} - \theta_{jt}), & \forall (i,j)\in\mathcal{L}, t \in \mathcal{T}, 
\end{dcases} 
\\
& \parbox{3cm}{\it \small Integrality Restrictions \& Variable \newline Bounds} 
\begin{dcases} 
x_{gt}, \, s_{gt}, \, z_{gt} \in \{ 0 , 1 \}, & \forall g \in \mathcal{G}, \, t \in \mathcal{T},
\\ 
G_{gt} \geq 0, & \forall g \in \mathcal{G}, \, t \in \mathcal{T},  
\\ 
-\pi \leq \theta_{jt} \leq \pi, ~ L_{jt} \geq 0, \, O_{jt} \geq 0 & \forall j \in \mathcal{B}, \, t \in \mathcal{T}, 
\end{dcases} 
\end{align}
\end{subequations}

In deterministic UC formulations, alternative representations of the above constraints were used to gain computational performance. For brevity, such representations are omitted from this paper. 

\subsection{Economic Dispatch}



\end{document}
