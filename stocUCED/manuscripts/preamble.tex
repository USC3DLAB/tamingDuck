\documentclass[11pt,a4paper]{article}
\usepackage[utf8]{inputenc}
\usepackage{amsmath}
\usepackage{amsthm}
\usepackage{amsfonts}
\usepackage{amssymb}
%\usepackage{algorithmic}
%\usepackage{algorithm}
\usepackage{hyperref}
\usepackage[ruled,linesnumbered,procnumbered]{algorithm2e}

\makeatletter
\let\original@algocf@latexcaption\algocf@latexcaption
\long\def\algocf@latexcaption#1[#2]{%
  \@ifundefined{NR@gettitle}{%
    \def\@currentlabelname{#2}%
  }{%
    \NR@gettitle{#2}%
  }%
  \original@algocf@latexcaption{#1}[{#2}]%
}
\makeatother

\usepackage{xcolor}		% for adding colors to the text

\usepackage[footnotesize, center]{caption}	
\usepackage{subcaption}
%\usepackage{subfigure}
\usepackage{booktabs}
\usepackage{multirow}
\usepackage{breakurl}	% fixes the problem of breaking long url into lines inside the bibliography (IMP: must be below the hyperref package)
\hypersetup{
    bookmarks=true,		% show bookmarks bar?
    colorlinks=true,		% false: boxed links; true: colored links
    linkcolor=blue,		% color of internal links
    citecolor=blue,		% color of links to bibliography
    filecolor=blue,		% color of file links
    urlcolor=blue,		% color of external links
    bookmarksopen=true,
    breaklinks=true,
}
%\usepackage{natbib}			% bibliography
%\setlength{\bibsep}{0pt}

\usepackage[title]{appendix}

\usepackage{longtable}

% Page Setup

% Margins
	% MS Word (Default):		[margin=0.98in]
	% MS Word (Narrow):		[margin=0.5in]
\usepackage[margin=2cm]{geometry}
\setlength{\parskip}{\medskipamount}

\usepackage{setspace}
%\onehalfspacing
\setstretch{1.15}

\usepackage{indentfirst}			% indent the first paragraph
\allowdisplaybreaks				% break if equations take too much vertical space

% Section Style
\makeatletter
\def\@seccntformat#1{\csname the#1\endcsname.\quad}		% put a fullstop after section numbers
\makeatother

% Optional packages
\usepackage{xfrac}	% nicer fractional symbols (e.g., \sfrac{1}{2})
\usepackage{fancyhdr}	% adding headers / footers

% MACROS

% mathematical
\newcommand{\mathbold}[1]{\boldsymbol{#1}}		% for adding bold greek letters
\newcommand{\binary}{\{ 0, 1 \} }

\newcommand{\parentheses}[1]{\left( #1 \right) }	% auto-size parentheses
\newcommand{\brackets}[1]{\left[ #1 \right] }		% auto-size brackets
\newcommand{\curly}[1]{\left\{ #1 \right\} }		% auto-size curly brackets

	% Nilay Noyan's commenting macros
	% commentbox (adds black comments in a frame box)
	% comment (adds red comments as a footnote)

\newcounter{commentcounter}
\setcounter{commentcounter}{1}

\setlength{\fboxsep}{5pt}
\setlength{\fboxrule}{0.1pt}

\newcommand{\commentbox}[1]{
% \hfill \newline \noindent
%	\framebox[\textwidth]{
%		\parbox{0.98\textwidth}{
%			\footnotesize{
%				\texttt{\textcolor{black}{(C.\arabic{commentcounter})~#1\hfill}}}}}
%	\addtocounter{commentcounter}{1} 
	}

\long\def\symbolfootnote[#1]#2{\begingroup\def\thefootnote{\fnsymbol{footnote}}\footnote[#1]{#2}\endgroup}

\newcommand{\comment}[2]{{\footnotesize\texttt{\textcolor{red}{(C.\arabic{commentcounter})}}\symbolfootnote[4]{\texttt{\textcolor{red}
        {(C.\arabic{commentcounter}) [#1]: ~#2}}}}\addtocounter{commentcounter}{1}}
%\newcommand{\comment}[2]{}

% theorems
% the subtheorem environment is used to generate theorem numbers 1a, 1b, etc.
% Source: http://tex.stackexchange.com/questions/43346/how-do-i-get-sub-numbering-for-theorems-theorem-1-a-theorem-1-b-theorem-2
\makeatletter
\newenvironment{subtheorem}[1]{%
  \def\subtheoremcounter{#1}%
  \refstepcounter{#1}%
  \protected@edef\theparentnumber{\csname the#1\endcsname}%
  \setcounter{parentnumber}{\value{#1}}%
  \setcounter{#1}{0}%
  \expandafter\def\csname the#1\endcsname{\theparentnumber\alph{#1}}%
  \ignorespaces
}{%
  \setcounter{\subtheoremcounter}{\value{parentnumber}}%
  \ignorespacesafterend
}
\makeatother
\newcounter{parentnumber}
% end of subtheorem environment

\newtheorem{theorem}{\bf Theorem}
\newtheorem{lemma}{\bf Lemma}
\newtheorem{proposition}{\bf Proposition}
\newtheorem{corollary}{\bf Corollary}
\newtheorem{definition}{\sc Definition}
\newtheorem{fact}{\bf Fact}
\newtheorem{claim}{\sc Claim}
\newtheorem{case}{\sc Case}
\newtheorem{observation}{\sc Observation}
\renewcommand{\qedsymbol}{\hfill \tiny$\blacksquare$}		% symbol for proof environment
\renewcommand{\proofname}{\textnormal{\textbf{Proof.}}}	% title in the proof environment

\newtheoremstyle{mytheoremstyle} % name
    {\topsep}                    % Space above
    {\topsep}                    % Space below
    {}                   % Body font
    {}                           % Indent amount
    {\scshape}                   % Theorem head font
    {.}                          % Punctuation after theorem head
    {.5em}                       % Space after theorem head
    {}  % Theorem head spec (can be left empty, meaning ‘normal’)
\theoremstyle{mytheoremstyle}
\newtheorem{example}{Example}

\newtheoremstyle{myassumptionstyle} % name
    {\smallskipamount}                    % Space above
    {0}                    % Space below
    {}                   % Body font
    {}                     	% Indent amount
    {\upshape}              % Theorem head font
    {.}                          	% Punctuation after theorem head
    {.5em}                      % Space after theorem head
    {}  				% Theorem head spec (can be left empty, meaning ‘normal’)
\theoremstyle{myassumptionstyle}
\newtheorem{assumption}{\bf A\ignorespaces}
\newtheorem{remark}{\bf Remark}

\DeclareMathOperator*{\argmin}{arg\,min} 
\DeclareMathOperator*{\argmax}{arg\,max} 


% narrative
\newcommand{\ie}{\textit{i.e.}}		% id est, that is to say
\newcommand{\ex}{\textit{ex.}}		% example
\newcommand{\eg}{\textit{e.g.}}	% exempli gratia, for the sake of example

\newcommand{\st}{\text{subject to:}\qquad}	% subject to
\newcommand{\mathbi}[1]{\boldsymbol{#1}}	% \boldsymbol{ any character } // makes both italic and bold

\newcommand{\cplex}{\texttt{CPLEX}}

\newcommand{\question}[1]{\vspace{\baselineskip}\noindent\pdfbookmark{Question #1}{Question #1}\textbf{\large{Question #1}} \normalsize\medskip\newline}
\newcommand{\qpart}[1]{\indent\textbf{#1)}}	% i.e., a), b), ...

\newcommand{\inlinecomment}[1]{{\color[rgb]{0.13,0.57,0.4} \textbf{#1}}}

% algorithmic
\newcommand{\np}{$\mathcal{NP}$}	% e.g., as in NP-hard

\usepackage{enumitem}	% for aligned descriptions
\usepackage{calc} 		% for aligned descriptions

\setenumerate{
itemsep=0pt,
partopsep=0pt,
parsep=0pt,
topsep=0pt,
labelindent=4pt,
font=\normalfont
}

\setdescription{
itemsep=0pt,
partopsep=0pt,
parsep=0pt,
topsep=0pt,
labelindent=4pt,
font=\normalfont
}

\setitemize{
itemsep=0pt,
partopsep=0pt,
parsep=0pt,
topsep=0pt,
labelindent=4pt,
font=\normalfont
}
